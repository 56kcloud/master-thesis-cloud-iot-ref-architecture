% ------------------------------------------------------------------------------
% A conclusion succinctly summarizes the most important results and thus represents the highlight of your Bachelor thesis.
%
% In the conclusion, you only mention information and conclusions that you have already mentioned in the course of your body text. New information is not mentioned here.
% The conclusion should have the following content:
% - Project summary
% - Comparison with the initial objectives
% - Encountered difficulties
% - Future perspectives
% ------------------------------------------------------------------------------

\opt{never}{\addbibresource{03-tail/bibliography.bib}} % to make citation found in most IDE

\chapter{Conclusions}
\label{chap:conclusions}

% -----------------------------------------------------------------------------
\section{Project summary}

% -- Your text goes here --
In summary, this project has resulted in the creation of an infrastructure that enables the fluid, automated and secure integration of \acrshort{iot} devices into an \gls{aws} \gls{cloud} environment. The reference architecture aims to simplify the integration of embedded systems within an \gls{aws} \gls{cloud_infrastructure} dedicated to the \acrshort{iot}. The devices used in this \acrshort{iot} network feature \gls{arm} architecture and are \nameref{sec:arm_systemready} certified. To improve this integration, a \acrshort{devops} approach has been put in place thanks to a continuous integration and distribution process orchestrated by GitHub Actions. The central tool for deploying the infrastructure is Pulumi, which enables \gls{aws} \gls{cloud_infrastructure} to be deployed on various environments. Python applications have been specially developed to demonstrate how integration works via \acrshort{mqtt} communication between devices and \gls{aws} \acrshort{iot}. In the \acrshort{ci}/\acrshort{cd} process, these applications are tested, published in \gls{aws} and then deployed to a fleet of devices. During these deployments, the applications run in Docker containers, ensuring greater portability. A custom \acrshort{os} image, based on Raspberry Pi \acrshort{os} Lite, has been developed, incorporating Docker for applications and \gls{aws} \acrshort{iot} Greengrass Core, which is essential for provisioning and integration.  This \acrshort{os} image is downloaded, flashed onto an SD card and inserted into an embedded system, which is automatically provisioned in a few minutes when it is first booted up. The successful integration of a first Raspberry Pi 4 has been extended to a second Raspberry Pi 4, demonstrating the scalability of the architecture. To guarantee network reliability, security mechanisms are built into the architecture, making communications within the \acrshort{iot} system even more robust and protected.

% -----------------------------------------------------------------------------
\section{Comparison with the initial objectives}

% -- Your text goes here --
The project was an overall success, achieving the main objective of designing a reference architecture to facilitate the deployment of a \gls{cloud_infrastructure} while automating the provisioning of a fleet of embedded systems. Now available as open source, this achievement provides a solid foundation for engineers looking for similar solutions. Detailed documentation accompanies the project to make it easier to understand and use.

The \nameref{subsec:cloudnative} approach and the \acrshort{devops} methodology, with continuous integration and continuous distribution, played an essential role in the development of the architecture. These choices reinforced the robustness of the architecture, simplifying its use and enabling the integration of missing elements specific to each solution.

The \gls{aws} \gls{cloud_infrastructure} is totally modular, offering the possibility of adding the additional services required for the product. The use of the Pulumi \acrshort{iac} tool, integrated into this architecture, facilitates this flexibility.

The integration of the embedded systems has been carried out in such a way as to facilitate the deployment, updating and interaction of the applications, while guaranteeing the security of the \acrshort{iot} network. The use of Docker, well accepted by the central \gls{aws} \acrshort{iot} Greengrass Core service, enables applications to be launched in containers, offering greater portability. What's more, Docker images are easily accessible in a dedicated registry in the \gls{cloud}. Data flows seamlessly between \acrshort{iot} devices and the \gls{aws} \gls{cloud}. It can be viewed from the \gls{aws} client console.

However, a challenge remains in terms of compatibility with embedded systems. Although \gls{arm} and \nameref{sec:arm_systemready} certified devices were taken into account, booting from the same \acrshort{os} image was only possible with the Raspberry Pi 4, restricting the scope of the architecture. The integration of the Packer tool nevertheless offers a solution for modifying the creation of the \acrshort{os} image, enabling the base distribution and its configuration to be changed.

In terms of project management, the use of established methodologies such as Scrum and \gls{kanban} ensured that deadlines were met. Some deviations were observed, notably in the confusion of milestones. The term "proof of concept" turned out to be the reference architecture itself, encompassing the entire project. Despite some disparities between the forecast and actual schedules, the renamed milestones accurately reflect the different phases of the project. One notable difference is the amount of time devoted to process automation, which is taking almost as long as the implementation of the \gls{cloud_infrastructure}. The detailed schedules can be consulted in the appendix (\ref{annexe_planning}).

% -----------------------------------------------------------------------------
\section{Encountered difficulties}

% -- Your text goes here --
In the course of this work, several challenges emerged. Firstly, the definition of the reference architecture. This proved difficult due to the vague nature of the term, which can vary from one domain to another. Positioning the components to be implemented to create a coherent reference architecture was complex and the overall vision was not always clear. However, by starting with the implementation of a few elements such as applications, it was possible to better define the objective to be achieved.

Next, the complexity of embedded systems compatibility was another difficulty to overcome. Creating an \acrshort{os} image compatible with various embedded systems proved to be a major challenge. The lack of time at the end of the project limited the possibility of finding a satisfactory solution.

On the whole, however, the tasks progressed without encountering any major obstacles.

% -----------------------------------------------------------------------------
\section{Future perspectives}

% -- Your text goes here --
Various horizons are opening up for the future development of this project. The reference architecture is now freely available to any developer wishing to adopt it, accompanied by comprehensive documentation to make it easier to understand and use.

The main prospect would be to invest in research and development to extend the architecture's compatibility with a wider range of embedded systems, taking into account the \gls{arm} architecture and \nameref{sec:arm_systemready} certification. One possible approach would be to modify the \acrshort{os} image creation process to design a generic version compatible with the entire range. If this is not possible, the creation of several \acrshort{os} images, listed by compatible device model, could be an alternative.

The integration of new \gls{cloud} services is a promising prospect. Continued evolution of the architecture would mean incorporating \gls{cloud} services and functionalities. Furthermore, applications developed for \acrshort{iot} devices could evolve to adapt to new emerging technologies, such as artificial intelligence.

Finally, the security of the \gls{cloud_infrastructure} could be strengthened. The use of additional \gls{aws} services could be explored to improve the monitoring of traffic on the infrastructure, in particular by integrating threat detection functionalities. This enhanced approach would help to ensure the security of data and connected \acrshort{iot} devices.