% ------------------------------------------------------------------------------
% A conclusion succinctly summarizes the most important results and thus represents the highlight of your Bachelor thesis.
%
% In the conclusion, you only mention information and conclusions that you have already mentioned in the course of your body text. New information is not mentioned here.
% The conclusion should have the following content:
% - Project summary
% - Comparison with the initial objectives
% - Encountered difficulties
% - Future perspectives
% ------------------------------------------------------------------------------

\opt{never}{\addbibresource{03-tail/bibliography.bib}} % to make citation found in most IDE

\chapter{Conclusions}
\label{chap:conclusions}

% -----------------------------------------------------------------------------
\section{Project summary}

% -- Your text goes here --
In summary, this project has resulted in the creation of an infrastructure that enables the fluid, automated and secure integration of \acrshort{iot} devices into an \gls{aws} \gls{cloud} environment. The reference architecture aims to simplify the integration of embedded systems within an \gls{aws} \gls{cloud_infrastructure} dedicated to the Internet of Things (\acrshort{iot}). The devices used in this \acrshort{iot} network feature \gls{arm} architecture and are \nameref{sec:arm_systemready} certified. To improve this integration, a \acrshort{devops} approach has been put in place thanks to a continuous integration and distribution process orchestrated by GitHub Actions. The central tool for deploying the infrastructure is Pulumi, which enables \gls{aws} \gls{cloud_infrastructure} to be deployed on various environments. Python applications have been specially developed to demonstrate how integration works via \acrshort{mqtt} communication between devices and \gls{aws} \acrshort{iot}. In the \acrshort{ci}/\acrshort{cd} process, these applications are tested, published in \gls{aws} and then deployed to a fleet of devices. During these deployments, the applications run in Docker containers, ensuring greater portability. A custom \acrshort{os} image, based on Raspberry Pi \acrshort{os} Lite, has been developed, incorporating Docker for applications and \gls{aws} \acrshort{iot} Greengrass Core, which is essential for provisioning and integration.  This \acrshort{os} image is downloaded, flashed onto an SD card and inserted into an embedded system, which is automatically provisioned in a few minutes when it is first booted up. The successful integration of a first Raspberry Pi 4 has been extended to a second Raspberry Pi 4, demonstrating the scalability of the architecture. To guarantee network reliability, security mechanisms are built into the architecture, making communications within the \acrshort{iot} system even more robust and protected.

% -----------------------------------------------------------------------------
\section{Comparison with the initial objectives}

% -- Your text goes here --


% -----------------------------------------------------------------------------
\section{Encountered difficulties}

% -- Your text goes here --


% -----------------------------------------------------------------------------
\section{Future perspectives}

% -- Your text goes here --
