% ------------------------------------------------------------------------------
% One Page
% - Problem
% - Objective(s)
% - Approach and method
% - Results, Solutions (This part should have the most prominence)
% - Conclusion
% ------------------------------------------------------------------------------

\chapter*{Abstract}
\addcontentsline{toc}{chapter}{Abstract} % adds an entry to the table of contents


% -- Your text goes here --
Cloud-Native IoT Reference Architecture with Arm SystemReady is an open-source project designed to meet the integration challenges between embedded systems and the cloud. This architecture facilitates the automatic provisioning of a cloud infrastructure and the integration of a fleet of embedded systems during their initial start-up. Designed for use with AWS, it focuses on core components while leveraging Arm processors and SystemReady certifications. The project aims to streamline collaboration between embedded systems engineers and cloud professionals, allowing them to focus on their end products. By following Cloud-Native best practice, the continuous integration and delivery tools guarantee a robust, functional architecture on an Arm-based embedded system that is SystemReady certified. The ultimate aim is to encourage the creation of a community around this architecture, enabling engineers to adopt it and apply it easily to their projects. In addition, a demonstration is included involving the deployment of a cloud infrastructure on AWS and the automatic configuration of cloud services for seamless interaction with an embedded system, plus data visualisation accessible via a minimal web interface.

\vspace{0.5cm}
\textbf{Key words:}
\Keywords
